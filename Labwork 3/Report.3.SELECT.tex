%----------------------------------------------------------------------------------------
%   PACKAGES AND OTHER DOCUMENT CONFIGURATIONS -DONT NEED TO TOUCH
%----------------------------------------------------------------------------------------
\documentclass{article}
\usepackage{fancyhdr} % Required for custom headers
% \usepackage{lastpage} % Required to determine the last page for the footer
\usepackage{extramarks} % Required for headers and footers
\usepackage[usenames,dvipsnames]{color} % Required for custom colors

\usepackage{subcaption}
\usepackage{graphicx} % Required to insert images
\usepackage{listings} % Required for insertion of code
\usepackage{color}
 
\definecolor{codegreen}{rgb}{0,0.6,0}
\definecolor{codegray}{rgb}{0.5,0.5,0.5}
\definecolor{codepurple}{rgb}{0.58,0,0.82}
\definecolor{backcolour}{rgb}{0.95,0.95,0.92}
 
\lstdefinestyle{mystyle}{
    backgroundcolor=\color{backcolour},   
    commentstyle=\color{codegreen},
    keywordstyle=\color{magenta},
    numberstyle=\tiny\color{codegray},
    stringstyle=\color{codepurple},
    basicstyle=\footnotesize,
    breakatwhitespace=false,         
    breaklines=true,                 
    captionpos=b,                    
    keepspaces=true,                 
    numbers=left,                    
    numbersep=5pt,                  
    showspaces=false,                
    showstringspaces=false,
    showtabs=false,                  
    tabsize=2
}
 
\lstset{style=mystyle}
\usepackage{courier} % Required for the courier font
% \usepackage{comment}
% \usepackage{multirow}
\usepackage{amsmath}
\usepackage[table,xcdraw]{xcolor}
\usepackage{geometry}
\usepackage{pdflscape}
\usepackage{pdfpages}
\usepackage{booktabs}
\usepackage{geometry}
\setlength{\heavyrulewidth}{1.5pt}
\setlength{\abovetopsep}{4pt}

% Margins
\topmargin=-0.45in
\evensidemargin=0in
\oddsidemargin=0in
\textwidth=6.5in
\textheight=9.0in
\headsep=0.25in

\linespread{1.1} % Line spacing

% Set up the header and footer
\pagestyle{fancy}
\lhead{\hmwkAuthorName} % Top left header
\chead{\hmwkClass\ : \hmwkTitle} % Top center head
\rhead{} %Nothing
\cfoot{} % Bottom center footer
\rfoot{\thepage} % Bottom right footer
\renewcommand\headrulewidth{0.4pt} % Size of the header rule
\renewcommand\footrulewidth{0.4pt} % Size of the footer rule

\setlength\parindent{0pt} % Removes all indentation from paragraphs

%----------------------------------------------------------------------------------------
%   DOCUMENT STRUCTURE COMMANDS
%   Skip this unless you know what you're doing
%----------------------------------------------------------------------------------------

% Header and footer for when a page split occurs within a problem environment
\newcommand{\enterProblemHeader}[1]{
\nobreak\extramarks{#1}{#1 continued on next page\ldots}\nobreak
\nobreak\extramarks{#1 (continued)}{#1 continued on next page\ldots}\nobreak
}

% Header and footer for when a page split occurs between problem environments
\newcommand{\exitProblemHeader}[1]{
\nobreak\extramarks{#1 (continued)}{#1 continued on next page\ldots}\nobreak
\nobreak\extramarks{#1}{}\nobreak
}

\setcounter{secnumdepth}{0} % Removes default section numbers
\newcounter{homeworkProblemCounter} % Creates a counter to keep track of the number of problems

\newcommand{\homeworkProblemName}{}

\newenvironment{homeworkProblem}[1][Problem \arabic{homeworkProblemCounter}]{ % Makes a new environment called homeworkProblem which takes 1 argument (custom name) but the default is "Problem #"
\stepcounter{homeworkProblemCounter} % Increase counter for number of problems

\renewcommand{\homeworkProblemName}{#1} % Assign \homeworkProblemName the name of the problem

\section{\homeworkProblemName} % Make a section in the document with the custom problem count

}

\newcommand{\problemAnswer}[1]{ % Defines the problem answer command with the content as the only argument
\noindent\framebox[\columnwidth][c]{\begin{minipage}{0.98\columnwidth}#1\end{minipage}} % Makes the box around the problem answer and puts the content inside
}

\newcommand{\homeworkSectionName}{}

\newenvironment{homeworkSection}[1]{ % New environment for sections within homework problems, takes 1 argument - the name of the section

\renewcommand{\homeworkSectionName}{#1} % Assign \homeworkSectionName to the name of the section from the environment argument
\subsection{\homeworkSectionName} % Make a subsection with the custom name of the subsection
}

\newcommand{\overbar}[1]{\mkern 1.5mu\overline{\mkern-1.5mu#1\mkern-1.5mu}\mkern 1.5mu}

%----------------------------------------------------------------------------------------
%   NAME AND CLASS SECTION ***SPECIFY PERSONAL DETAILS HERE***
%----------------------------------------------------------------------------------------

\newcommand{\hmwkTitle}{Labwork 3} % Assignment title
% \newcommand{\hmwkDueDate}{Wednesday,\ January\ 18,\ 2017} % Due date
\newcommand{\hmwkClass}{Information\ Systems} % Course/class
% \newcommand{\hmwkClassTime}{10:05AM} % Class/lecture time
\newcommand{\hmwkClassInstructor}{Nghiem Thi Phuong} % Teacher/lecturer
\newcommand{\hmwkAuthorName}{Nguyen Duc Tung} % Your name

%----------------------------------------------------------------------------------------
%   TITLE PAGE
%----------------------------------------------------------------------------------------

\title{
\vspace{2in}
\textmd{\textbf{\hmwkClass:\ \hmwkTitle}}\\
% \normalsize\vspace{0.1in}\small{Due\ on\ \hmwkDueDate}\\
\vspace{0.1in}\large{\textit{\hmwkClassInstructor}}
\vspace{3in}
}

\author{\textbf{\hmwkAuthorName}}
\date{} % Insert date here if you want it to appear below your name

%----------------------------------------------------------------------------------------

%%TITLE PAGE INSERT--------
\begin{document}
\maketitle
\clearpage
%%-------------------

%%Problem 1_________________________
\begin{homeworkProblem}[Labwork 3]

\begin{homeworkSection}{All info of all employees}
\begin{lstlisting}[language=SQL]
SELECT * FROM EMPLOYEES;
\end{lstlisting}
Output:
\begin{lstlisting}
...
| 499996 | 1953-03-07 | Zito           | Baaz             | M      | 1990-09-27 |
| 499997 | 1961-08-03 | Berhard        | Lenart           | M      | 1986-04-21 |
| 499998 | 1956-09-05 | Patricia       | Breugel          | M      | 1993-10-13 |
| 499999 | 1958-05-01 | Sachin         | Tsukuda          | M      | 1997-11-30 |
+--------+------------+----------------+------------------+--------+------------+
300024 rows in set (0.26 sec)
\end{lstlisting}
\end{homeworkSection}

\begin{homeworkSection}{All info of all departments}
\begin{lstlisting}[language=SQL]
SELECT * FROM DEPARTMENTS;
\end{lstlisting}
Output:
\begin{lstlisting}
+---------+--------------------+
| dept_no | dept_name          |
+---------+--------------------+
| d009    | Customer Service   |
| d005    | Development        |
| d002    | Finance            |
| d003    | Human Resources    |
| d001    | Marketing          |
| d004    | Production         |
| d006    | Quality Management |
| d008    | Research           |
| d007    | Sales              |
+---------+--------------------+
9 rows in set (0.00 sec)
\end{lstlisting}
\end{homeworkSection}

\begin{homeworkSection}{Full names of all emplyees}
\begin{lstlisting}[language=SQL]
SELECT CONCAT(FIRST_NAME, " ", LAST_NAME) AS FULL_NAME FROM EMPLOYEES;
\end{lstlisting}
Output:
\begin{lstlisting}
...
| Zito Baaz                      |
| Berhard Lenart                 |
| Patricia Breugel               |
| Sachin Tsukuda                 |
+--------------------------------+
300024 rows in set (0.21 sec)
\end{lstlisting}
\end{homeworkSection}

\begin{homeworkSection}{Names of all departments}
\begin{lstlisting}[language=SQL]
SELECT DEPT_NAME FROM DEPARTMENTS;
\end{lstlisting}
Output:
\begin{lstlisting}
+--------------------+
| DEPT_NAME          |
+--------------------+
| Customer Service   |
| Development        |
| Finance            |
| Human Resources    |
| Marketing          |
| Production         |
| Quality Management |
| Research           |
| Sales              |
+--------------------+
9 rows in set (0.00 sec)
\end{lstlisting}
\end{homeworkSection}

\begin{homeworkSection}{Full names of employees working in "Sales" department}
\begin{lstlisting}[language=SQL]
SELECT CONCAT(FIRST_NAME, " ", LAST_NAME) AS FULL_NAME FROM EMPLOYEES
    JOIN DEPT_EMP ON EMPLOYEES.EMP_NO = DEPT_EMP.EMP_NO
    JOIN DEPARTMENTS ON DEPARTMENTS.DEPT_NO = DEPT_EMP.DEPT_NO
WHERE DEPT_NAME = 'Sales';
\end{lstlisting}
Output:
\begin{lstlisting}
...
| Gino Usery                    |
| Nathan Ranta                  |
| Rimli Dusink                  |
| Bangqing Kleiser              |
+-------------------------------+
52245 rows in set (0.18 sec)
\end{lstlisting}
\end{homeworkSection}

\begin{homeworkSection}{Full names of male employees working in "Finance" department }
\begin{lstlisting}[language=SQL]
SELECT CONCAT(FIRST_NAME, " ", LAST_NAME) AS FULL_NAME FROM EMPLOYEES
    JOIN DEPT_EMP ON EMPLOYEES.EMP_NO = DEPT_EMP.EMP_NO
    JOIN DEPARTMENTS ON DEPARTMENTS.DEPT_NO = DEPT_EMP.DEPT_NO
WHERE DEPT_NAME = 'Finance' AND GENDER = "M";
\end{lstlisting}
Output:
\begin{lstlisting}
...
| Shaz Bierbaum                |
| Masali Chorvat               |
| Keiichiro Lindqvist          |
| Patricia Breugel             |
+------------------------------+
10331 rows in set (0.04 sec)
\end{lstlisting}
\end{homeworkSection}

\begin{homeworkSection}{Salaries of female employees working in "Marketing" department}
\begin{lstlisting}[language=SQL]
SELECT SALARY FROM EMPLOYEES
    JOIN SALARIES ON SALARIES.EMP_NO = EMPLOYEES.EMP_NO
    JOIN DEPT_EMP ON EMPLOYEES.EMP_NO = DEPT_EMP.EMP_NO
    JOIN DEPARTMENTS ON DEPARTMENTS.DEPT_NO = DEPT_EMP.DEPT_NO
WHERE DEPT_NAME = 'Marketing' AND GENDER = "F";
\end{lstlisting}
Output:
\begin{lstlisting}
...
|  73574 |
|  76708 |
|  79327 |
|  80389 |
+--------+
74996 rows in set (0.34 sec)
\end{lstlisting}
\end{homeworkSection}

\begin{homeworkSection}{Full names of employees who have the same last name as their manager}
\begin{lstlisting}[language=SQL]
SELECT CONCAT(EMPLOYEES.FIRST_NAME, " ", EMPLOYEES.LAST_NAME) AS FULL_NAME FROM EMPLOYEES
    JOIN DEPT_EMP ON EMPLOYEES.EMP_NO = DEPT_EMP.EMP_NO
    JOIN DEPT_MANAGER ON DEPT_EMP.DEPT_NO = DEPT_MANAGER.DEPT_NO
    JOIN EMPLOYEES AS MANAGER ON DEPT_MANAGER.EMP_NO = MANAGER.EMP_NO
WHERE EMPLOYEES.LAST_NAME = MANAGER.LAST_NAME;
\end{lstlisting}
Output:
\begin{lstlisting}
...
| Shen Weedman           |
| Dinkar Weedman         |
| Jaques Weedman         |
| Marc Weedman           |
+------------------------+
572 rows in set (1.31 sec)
\end{lstlisting}
\end{homeworkSection}

\begin{homeworkSection}{Full names of managers who have been doing the job at least twice}
\begin{lstlisting}[language=SQL]
SELECT CONCAT(FIRST_NAME, " ", LAST_NAME) AS FULL_NAME FROM EMPLOYEES
    JOIN DEPT_MANAGER ON DEPT_MANAGER.EMP_NO = EMPLOYEES.EMP_NO
GROUP BY EMPLOYEES.EMP_NO HAVING COUNT(*) >= 2;
\end{lstlisting}
Output:
\begin{lstlisting}
Empty set (0.01 sec)
\end{lstlisting}
\end{homeworkSection}

\begin{homeworkSection}{Full names of employees who was paid more than \$100000}
\begin{lstlisting}[language=SQL]
SELECT DISTINCT CONCAT(FIRST_NAME, " ", LAST_NAME) AS FULL_NAME FROM EMPLOYEES
    JOIN SALARIES ON SALARIES.EMP_NO = EMPLOYEES.EMP_NO
WHERE SALARY > 100000;
\end{lstlisting}
Output:
\begin{lstlisting}
...
| Guozhong Felder             |
| Gino Usery                  |
| Nathan Ranta                |
| Bangqing Kleiser            |
+-----------------------------+
18925 rows in set (9.46 sec)
\end{lstlisting}
\end{homeworkSection}

\begin{homeworkSection}{Names of all departments that have employees paid more than \$100000}
\begin{lstlisting}[language=SQL]
SELECT DISTINCT DEPT_NAME FROM DEPARTMENTS
    JOIN DEPT_EMP ON DEPARTMENTS.DEPT_NO = DEPT_EMP.DEPT_NO
    JOIN SALARIES ON SALARIES.EMP_NO = DEPT_EMP.EMP_NO
WHERE SALARY > 100000;
\end{lstlisting}
Output:
\begin{lstlisting}
+--------------------+
| DEPT_NAME          |
+--------------------+
| Customer Service   |
| Development        |
| Finance            |
| Human Resources    |
| Marketing          |
| Production         |
| Quality Management |
| Research           |
| Sales              |
+--------------------+
9 rows in set (0.02 sec)
\end{lstlisting}
\end{homeworkSection}

\end{homeworkProblem}
\clearpage
%%_______________________________

%%APPENDICES____________________
%%\clearpage
%%______________________________
\end{document}