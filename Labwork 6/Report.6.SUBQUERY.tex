%----------------------------------------------------------------------------------------
%   PACKAGES AND OTHER DOCUMENT CONFIGURATIONS -DONT NEED TO TOUCH
%----------------------------------------------------------------------------------------
\documentclass{article}
\usepackage{fancyhdr} % Required for custom headers
% \usepackage{lastpage} % Required to determine the last page for the footer
\usepackage{extramarks} % Required for headers and footers
\usepackage[usenames,dvipsnames]{color} % Required for custom colors

\usepackage{subcaption}
\usepackage{graphicx} % Required to insert images
\usepackage{listings} % Required for insertion of code
\usepackage{color}
 
\definecolor{codegreen}{rgb}{0,0.6,0}
\definecolor{codegray}{rgb}{0.5,0.5,0.5}
\definecolor{codepurple}{rgb}{0.58,0,0.82}
\definecolor{backcolour}{rgb}{0.95,0.95,0.92}
 
\lstdefinestyle{mystyle}{
    backgroundcolor=\color{backcolour},   
    commentstyle=\color{codegreen},
    keywordstyle=\color{magenta},
    numberstyle=\tiny\color{codegray},
    stringstyle=\color{codepurple},
    basicstyle=\footnotesize,
    breakatwhitespace=false,         
    breaklines=true,                 
    captionpos=b,                    
    keepspaces=true,                 
    numbers=left,                    
    numbersep=5pt,                  
    showspaces=false,                
    showstringspaces=false,
    showtabs=false,                  
    tabsize=2
}
 
\lstset{style=mystyle}
\usepackage{courier} % Required for the courier font
% \usepackage{comment}
% \usepackage{multirow}
\usepackage{amsmath}
\usepackage[table,xcdraw]{xcolor}
\usepackage{geometry}
\usepackage{pdflscape}
\usepackage{pdfpages}
\usepackage{booktabs}
\usepackage{geometry}
\setlength{\heavyrulewidth}{1.5pt}
\setlength{\abovetopsep}{4pt}

% Margins
\topmargin=-0.45in
\evensidemargin=0in
\oddsidemargin=0in
\textwidth=6.5in
\textheight=9.0in
\headsep=0.25in

\linespread{1.1} % Line spacing

% Set up the header and footer
\pagestyle{fancy}
\lhead{\hmwkAuthorName} % Top left header
\chead{\hmwkClass\ : \hmwkTitle} % Top center head
\rhead{} %Nothing
\cfoot{} % Bottom center footer
\rfoot{\thepage} % Bottom right footer
\renewcommand\headrulewidth{0.4pt} % Size of the header rule
\renewcommand\footrulewidth{0.4pt} % Size of the footer rule

\setlength\parindent{0pt} % Removes all indentation from paragraphs

%----------------------------------------------------------------------------------------
%   DOCUMENT STRUCTURE COMMANDS
%   Skip this unless you know what you're doing
%----------------------------------------------------------------------------------------

% Header and footer for when a page split occurs within a problem environment
\newcommand{\enterProblemHeader}[1]{
\nobreak\extramarks{#1}{#1 continued on next page\ldots}\nobreak
\nobreak\extramarks{#1 (continued)}{#1 continued on next page\ldots}\nobreak
}

% Header and footer for when a page split occurs between problem environments
\newcommand{\exitProblemHeader}[1]{
\nobreak\extramarks{#1 (continued)}{#1 continued on next page\ldots}\nobreak
\nobreak\extramarks{#1}{}\nobreak
}

\setcounter{secnumdepth}{0} % Removes default section numbers
\newcounter{homeworkProblemCounter} % Creates a counter to keep track of the number of problems

\newcommand{\homeworkProblemName}{}

\newenvironment{homeworkProblem}[1][Problem \arabic{homeworkProblemCounter}]{ % Makes a new environment called homeworkProblem which takes 1 argument (custom name) but the default is "Problem #"
\stepcounter{homeworkProblemCounter} % Increase counter for number of problems

\renewcommand{\homeworkProblemName}{#1} % Assign \homeworkProblemName the name of the problem

\section{\homeworkProblemName} % Make a section in the document with the custom problem count

}

\newcommand{\problemAnswer}[1]{ % Defines the problem answer command with the content as the only argument
\noindent\framebox[\columnwidth][c]{\begin{minipage}{0.98\columnwidth}#1\end{minipage}} % Makes the box around the problem answer and puts the content inside
}

\newcommand{\homeworkSectionName}{}

\newenvironment{homeworkSection}[1]{ % New environment for sections within homework problems, takes 1 argument - the name of the section

\renewcommand{\homeworkSectionName}{#1} % Assign \homeworkSectionName to the name of the section from the environment argument
\subsection{\homeworkSectionName} % Make a subsection with the custom name of the subsection
}

\newcommand{\overbar}[1]{\mkern 1.5mu\overline{\mkern-1.5mu#1\mkern-1.5mu}\mkern 1.5mu}

%----------------------------------------------------------------------------------------
%   NAME AND CLASS SECTION ***SPECIFY PERSONAL DETAILS HERE***
%----------------------------------------------------------------------------------------

\newcommand{\hmwkTitle}{Labwork 6} % Assignment title
% \newcommand{\hmwkDueDate}{Wednesday,\ January\ 18,\ 2017} % Due date
\newcommand{\hmwkClass}{Information\ Systems} % Course/class
% \newcommand{\hmwkClassTime}{10:05AM} % Class/lecture time
\newcommand{\hmwkClassInstructor}{Nghiem Thi Phuong} % Teacher/lecturer
\newcommand{\hmwkAuthorName}{Nguyen Duc Tung} % Your name

%----------------------------------------------------------------------------------------
%   TITLE PAGE
%----------------------------------------------------------------------------------------

\title{
\vspace{2in}
\textmd{\textbf{\hmwkClass:\ \hmwkTitle}}\\
% \normalsize\vspace{0.1in}\small{Due\ on\ \hmwkDueDate}\\
\vspace{0.1in}\large{\textit{\hmwkClassInstructor}}
\vspace{3in}
}

\author{\textbf{\hmwkAuthorName}}
\date{} % Insert date here if you want it to appear below your name

%----------------------------------------------------------------------------------------

%%TITLE PAGE INSERT--------
\begin{document}
\maketitle
\clearpage
%%-------------------

%%Problem 1_________________________
\begin{homeworkProblem}[Labwork 6]

\begin{homeworkSection}{Who have the same name as the managers of the “Finance” department?}
\begin{lstlisting}[language=SQL]
SELECT * FROM EMPLOYEES WHERE CONCAT(FIRST_NAME, " ", LAST_NAME) IN (
    SELECT CONCAT(FIRST_NAME, " ", LAST_NAME)
    FROM EMPLOYEES JOIN DEPT_MANAGER ON EMPLOYEES.EMP_NO = DEPT_MANAGER.EMP_NO JOIN DEPARTMENTS ON DEPT_MANAGER.DEPT_NO = DEPARTMENTS.DEPT_NO
    WHERE DEPT_NAME = "Finance");
\end{lstlisting}
Output:
\begin{lstlisting}
+--------+------------+------------+------------+--------+------------+
| emp_no | birth_date | first_name | last_name  | gender | hire_date  |
+--------+------------+------------+------------+--------+------------+
| 110085 | 1959-10-28 | Ebru       | Alpin      | M      | 1985-01-01 |
| 110114 | 1957-03-28 | Isamu      | Legleitner | F      | 1985-01-14 |
+--------+------------+------------+------------+--------+------------+
2 rows in set (0.32 sec)
\end{lstlisting}
\end{homeworkSection}

\begin{homeworkSection}{Who in the “Production” department were hired after the promotion of the last manager in that department?}
\begin{lstlisting}[language=SQL]
SELECT EMPLOYEES.EMP_NO, CONCAT(FIRST_NAME, " ", LAST_NAME), HIRE_DATE AS FULL_NAME
FROM EMPLOYEES JOIN DEPT_EMP ON EMPLOYEES.EMP_NO = DEPT_EMP.EMP_NO
WHERE DEPT_NO = (SELECT DEPT_NO FROM DEPARTMENTS WHERE DEPT_NAME = "Production")
AND HIRE_DATE >
    (SELECT MAX(FROM_DATE) FROM DEPT_MANAGER JOIN DEPARTMENTS ON DEPT_MANAGER.DEPT_NO = DEPARTMENTS.DEPT_NO
    WHERE DEPT_NAME = "Production");
\end{lstlisting}
Output:
\begin{lstlisting}
| 499856 | Yoshinari Theuretzbacher           | 1997-05-17 |
| 499916 | Florina Cusworth                   | 1997-05-18 |
| 499993 | DeForest Mullainathan              | 1997-04-07 |
| 499999 | Sachin Tsukuda                     | 1997-11-30 |
+--------+------------------------------------+------------+
3758 rows in set (0.13 sec)
\end{lstlisting}
\end{homeworkSection}

\begin{homeworkSection}{Find the average salary of each department, from highest to lowest.}
\begin{lstlisting}[language=SQL]
SELECT DEPT_NO, AVG(AVG_EMP) AS AVG_DEPT FROM DEPT_EMP JOIN
    (SELECT EMP_NO, AVG(SALARY) AS AVG_EMP FROM SALARIES GROUP BY EMP_NO) AS AVG_SALARY_EMP
ON DEPT_EMP.EMP_NO = AVG_SALARY_EMP.EMP_NO
GROUP BY DEPT_NO ORDER BY AVG_DEPT DESC;
\end{lstlisting}
Output:
\begin{lstlisting}
+---------+----------------+
| DEPT_NO | AVG_DEPT       |
+---------+----------------+
| d007    | 78313.22247361 |
| d001    | 69541.61771136 |
| d002    | 68061.43501801 |
| d008    | 57322.03105659 |
| d004    | 57253.31382027 |
| d005    | 57152.20845497 |
| d009    | 56480.08591880 |
| d006    | 54892.93507273 |
| d003    | 53214.29085744 |
+---------+----------------+
9 rows in set (3.02 sec)
\end{lstlisting}
\end{homeworkSection}

\begin{homeworkSection}{Find the average salary for each type of Engineer, from highest to lowest.}
\begin{lstlisting}[language=SQL]
SELECT TITLE, AVG(AVG_EMP) AS AVG_TITLE FROM TITLES JOIN
    (SELECT EMP_NO, AVG(SALARY) AS AVG_EMP FROM SALARIES GROUP BY EMP_NO) AS AVG_SALARY_EMP
ON TITLES.EMP_NO = AVG_SALARY_EMP.EMP_NO
WHERE TITLE LIKE '%Engineer%'
GROUP BY TITLE ORDER BY AVG_TITLE DESC;
\end{lstlisting}
Output:
\begin{lstlisting}
+--------------------+----------------+
| TITLE              | AVG_TITLE      |
+--------------------+----------------+
| Senior Engineer    | 59144.76835191 |
| Engineer           | 57244.45845623 |
| Assistant Engineer | 56963.53043254 |
+--------------------+----------------+
3 rows in set (3.15 sec)
\end{lstlisting}
\end{homeworkSection}

\end{homeworkProblem}
\clearpage
%%_______________________________

%%APPENDICES____________________
%%\clearpage
%%______________________________
\end{document}