%----------------------------------------------------------------------------------------
%   PACKAGES AND OTHER DOCUMENT CONFIGURATIONS -DONT NEED TO TOUCH
%----------------------------------------------------------------------------------------
\documentclass{article}
\usepackage{fancyhdr} % Required for custom headers
% \usepackage{lastpage} % Required to determine the last page for the footer
\usepackage{extramarks} % Required for headers and footers
\usepackage[usenames,dvipsnames]{color} % Required for custom colors

\usepackage{subcaption}
\usepackage{graphicx} % Required to insert images
\usepackage{listings} % Required for insertion of code
\usepackage{color}
 
\definecolor{codegreen}{rgb}{0,0.6,0}
\definecolor{codegray}{rgb}{0.5,0.5,0.5}
\definecolor{codepurple}{rgb}{0.58,0,0.82}
\definecolor{backcolour}{rgb}{0.95,0.95,0.92}
 
\lstdefinestyle{mystyle}{
    backgroundcolor=\color{backcolour},   
    commentstyle=\color{codegreen},
    keywordstyle=\color{magenta},
    numberstyle=\tiny\color{codegray},
    stringstyle=\color{codepurple},
    basicstyle=\footnotesize,
    breakatwhitespace=false,         
    breaklines=true,                 
    captionpos=b,                    
    keepspaces=true,                 
    numbers=left,                    
    numbersep=5pt,                  
    showspaces=false,                
    showstringspaces=false,
    showtabs=false,                  
    tabsize=2
}
 
\lstset{style=mystyle}
\usepackage{courier} % Required for the courier font
% \usepackage{comment}
% \usepackage{multirow}
\usepackage{amsmath}
\usepackage[table,xcdraw]{xcolor}
\usepackage{geometry}
\usepackage{pdflscape}
\usepackage{pdfpages}
\usepackage{booktabs}
\usepackage{geometry}
\setlength{\heavyrulewidth}{1.5pt}
\setlength{\abovetopsep}{4pt}

% Margins
\topmargin=-0.45in
\evensidemargin=0in
\oddsidemargin=0in
\textwidth=6.5in
\textheight=9.0in
\headsep=0.25in

\linespread{1.1} % Line spacing

% Set up the header and footer
\pagestyle{fancy}
\lhead{\hmwkAuthorName} % Top left header
\chead{\hmwkClass\ : \hmwkTitle} % Top center head
\rhead{} %Nothing
\cfoot{} % Bottom center footer
\rfoot{\thepage} % Bottom right footer
\renewcommand\headrulewidth{0.4pt} % Size of the header rule
\renewcommand\footrulewidth{0.4pt} % Size of the footer rule

\setlength\parindent{0pt} % Removes all indentation from paragraphs

%----------------------------------------------------------------------------------------
%   DOCUMENT STRUCTURE COMMANDS
%   Skip this unless you know what you're doing
%----------------------------------------------------------------------------------------

% Header and footer for when a page split occurs within a problem environment
\newcommand{\enterProblemHeader}[1]{
\nobreak\extramarks{#1}{#1 continued on next page\ldots}\nobreak
\nobreak\extramarks{#1 (continued)}{#1 continued on next page\ldots}\nobreak
}

% Header and footer for when a page split occurs between problem environments
\newcommand{\exitProblemHeader}[1]{
\nobreak\extramarks{#1 (continued)}{#1 continued on next page\ldots}\nobreak
\nobreak\extramarks{#1}{}\nobreak
}

\setcounter{secnumdepth}{0} % Removes default section numbers
\newcounter{homeworkProblemCounter} % Creates a counter to keep track of the number of problems

\newcommand{\homeworkProblemName}{}

\newenvironment{homeworkProblem}[1][Problem \arabic{homeworkProblemCounter}]{ % Makes a new environment called homeworkProblem which takes 1 argument (custom name) but the default is "Problem #"
\stepcounter{homeworkProblemCounter} % Increase counter for number of problems

\renewcommand{\homeworkProblemName}{#1} % Assign \homeworkProblemName the name of the problem

\section{\homeworkProblemName} % Make a section in the document with the custom problem count

}

\newcommand{\problemAnswer}[1]{ % Defines the problem answer command with the content as the only argument
\noindent\framebox[\columnwidth][c]{\begin{minipage}{0.98\columnwidth}#1\end{minipage}} % Makes the box around the problem answer and puts the content inside
}

\newcommand{\homeworkSectionName}{}

\newenvironment{homeworkSection}[1]{ % New environment for sections within homework problems, takes 1 argument - the name of the section

\renewcommand{\homeworkSectionName}{#1} % Assign \homeworkSectionName to the name of the section from the environment argument
\subsection{\homeworkSectionName} % Make a subsection with the custom name of the subsection
}

\newcommand{\overbar}[1]{\mkern 1.5mu\overline{\mkern-1.5mu#1\mkern-1.5mu}\mkern 1.5mu}

%----------------------------------------------------------------------------------------
%   NAME AND CLASS SECTION ***SPECIFY PERSONAL DETAILS HERE***
%----------------------------------------------------------------------------------------

\newcommand{\hmwkTitle}{Labwork 5} % Assignment title
% \newcommand{\hmwkDueDate}{Wednesday,\ January\ 18,\ 2017} % Due date
\newcommand{\hmwkClass}{Information\ Systems} % Course/class
% \newcommand{\hmwkClassTime}{10:05AM} % Class/lecture time
\newcommand{\hmwkClassInstructor}{Nghiem Thi Phuong} % Teacher/lecturer
\newcommand{\hmwkAuthorName}{Nguyen Duc Tung} % Your name

%----------------------------------------------------------------------------------------
%   TITLE PAGE
%----------------------------------------------------------------------------------------

\title{
\vspace{2in}
\textmd{\textbf{\hmwkClass:\ \hmwkTitle}}\\
% \normalsize\vspace{0.1in}\small{Due\ on\ \hmwkDueDate}\\
\vspace{0.1in}\large{\textit{\hmwkClassInstructor}}
\vspace{3in}
}

\author{\textbf{\hmwkAuthorName}}
\date{} % Insert date here if you want it to appear below your name

%----------------------------------------------------------------------------------------

%%TITLE PAGE INSERT--------
\begin{document}
\maketitle
\clearpage
%%-------------------

%%Problem 1_________________________
\begin{homeworkProblem}[Labwork 5]

\begin{homeworkSection}{What is the average salary of each employee?}
\begin{lstlisting}[language=SQL]
SELECT EMP_NO, AVG(SALARY) FROM SALARIES GROUP BY EMP_NO;
\end{lstlisting}
Output:
\begin{lstlisting}
...
| 499996 |  63134.1429 |
| 499997 |  66475.8667 |
| 499998 |  46665.5556 |
| 499999 |  70625.0000 |
+--------+-------------+
300024 rows in set (2.18 sec)
\end{lstlisting}
\end{homeworkSection}

\begin{homeworkSection}{How much was each employee paid in total?}
\begin{lstlisting}[language=SQL]
SELECT EMP_NO, SUM(SALARY) FROM SALARIES GROUP BY EMP_NO;
\end{lstlisting}
Output:
\begin{lstlisting}
...
| 499996 |      441939 |
| 499997 |      997138 |
| 499998 |      419990 |
| 499999 |      353125 |
+--------+-------------+
300024 rows in set (2.18 sec)
\end{lstlisting}
\end{homeworkSection}

\begin{homeworkSection}{Minimum, maximum and total salaries of each department?}
\begin{lstlisting}[language=SQL]
SELECT DEPT_NO, MIN(SALARY), MAX(SALARY), SUM(SALARY)
FROM DEPT_EMP JOIN SALARIES ON DEPT_EMP.EMP_NO = SALARIES.EMP_NO
GROUP BY DEPT_NO;
\end{lstlisting}
Output:
\begin{lstlisting}
+---------+-------------+-------------+-------------+
| DEPT_NO | MIN(SALARY) | MAX(SALARY) | SUM(SALARY) |
+---------+-------------+-------------+-------------+
| d001    |       39127 |      145128 | 13725425266 |
| d002    |       38812 |      142395 | 11650834677 |
| d003    |       38735 |      141953 |  9363811425 |
| d004    |       38623 |      138273 | 41554438942 |
| d005    |       38849 |      144434 | 48179456393 |
| d006    |       38786 |      132103 | 10865203635 |
| d007    |       39169 |      158220 | 40030089342 |
| d008    |       38851 |      130211 | 11969730427 |
| d009    |       38836 |      144866 | 13143639841 |
+---------+-------------+-------------+-------------+
9 rows in set (11.31 sec)
\end{lstlisting}
\end{homeworkSection}

\begin{homeworkSection}{Which departments have paid more than 20 Billion dollars for their employees?}
\begin{lstlisting}[language=SQL]
SELECT DEPT_NO, SUM(SALARY) AS TOTAL
FROM DEPT_EMP JOIN SALARIES ON DEPT_EMP.EMP_NO = SALARIES.EMP_NO
GROUP BY DEPT_NO HAVING TOTAL > 20000000000;
\end{lstlisting}
Output:
\begin{lstlisting}
+---------+-------------+
| DEPT_NO | TOTAL       |
+---------+-------------+
| d004    | 41554438942 |
| d005    | 48179456393 |
| d007    | 40030089342 |
+---------+-------------+
3 rows in set (11.35 sec)
\end{lstlisting}
\end{homeworkSection}

\begin{homeworkSection}{Total salaries of male employees in each department?}
\begin{lstlisting}[language=SQL]
SELECT DEPT_NO, SUM(SALARY) FROM DEPT_EMP
    JOIN EMPLOYEES ON DEPT_EMP.EMP_NO = EMPLOYEES.EMP_NO AND GENDER = 'M'
    JOIN SALARIES ON DEPT_EMP.EMP_NO = SALARIES.EMP_NO
GROUP BY DEPT_NO;
\end{lstlisting}
Output:
\begin{lstlisting}
+---------+-------------+
| DEPT_NO | SUM(SALARY) |
+---------+-------------+
| d001    |  8352455367 |
| d002    |  6940675318 |
| d003    |  5619533959 |
| d004    | 24873889749 |
| d005    | 28961357095 |
| d006    |  6472073211 |
| d007    | 24051537190 |
| d008    |  7174578852 |
| d009    |  7865587983 |
+---------+-------------+
9 rows in set (9.96 sec)
\end{lstlisting}
\end{homeworkSection}

\begin{homeworkSection}{Total salaries of department managers for each department, from highest to lowest?}
\begin{lstlisting}[language=SQL]
SELECT DEPT_NO, SUM(SALARY) AS TOTAL FROM DEPT_MANAGER
    JOIN SALARIES ON DEPT_MANAGER.EMP_NO = SALARIES.EMP_NO
GROUP BY DEPT_NO ORDER BY TOTAL DESC;
\end{lstlisting}
Output:
\begin{lstlisting}
+---------+---------+
| DEPT_NO | TOTAL   |
+---------+---------+
| d006    | 4162118 |
| d004    | 3374004 |
| d009    | 3187661 |
| d001    | 3093009 |
| d007    | 2915118 |
| d008    | 2558661 |
| d002    | 2549372 |
| d003    | 2098298 |
| d005    | 2028376 |
+---------+---------+
9 rows in set (0.02 sec)
\end{lstlisting}
\end{homeworkSection}

\end{homeworkProblem}
\clearpage
%%_______________________________

%%APPENDICES____________________
%%\clearpage
%%______________________________
\end{document}